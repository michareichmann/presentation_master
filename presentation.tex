\documentclass{beamer}
\usepackage[british]{babel}
%\usepackage{beamerarticle}
\usepackage[latin1]{inputenc}
\usepackage{amsmath,amsfonts,amssymb}
\usepackage{upgreek}
\newcommand{\as}{\\[14pt]}
\newcommand{\s}{\\[7pt]}
\newcommand{\no}{\noindent}
\newcommand{\ka}{\hspace*{0.5cm}}
\newcommand{\ma}{\hspace*{1cm}}
\newcommand{\ga}{\hspace*{1.5cm}}
\newcommand{\li}{\left|}
\newcommand{\re}{\right|}
\newcommand{\const}{\text{const.}}
\newcommand{\z}{\text}
\usepackage{pgfpages}
\usepackage[version=3]{mhchem}
\usepackage{lmodern}
\usepackage{graphicx}
\usepackage{multicol}
% \usepackage{uarial}
\usetheme{Boadilla}
\usecolortheme{beaver}
%\useinnertheme{circles}
\useoutertheme{miniframes}
%\setbeamercovered{transparent}
\beamertemplatenavigationsymbolsempty
%\setbeamertemplate{footline}[frame number]
\makeindex
\title[Master Thesis (Status)]{Status of the master thesis}
\author[M. Reichmann]{Michael Reichmann}
\institute[\textbf{\textit{ETH}}\scalebox{.6}{\textit{Z\"{u}rich}}]{Swiss Federal Institute of Technology Zurich / Low Energy Particle Physics}

% ============================
% START
% ============================
\begin{document}
% ============================
% TITLE PAGE
% ============================
\begin{frame}
	\includegraphics[width=\linewidth]{Pics/ETH-Zuerich}
	\vspace*{.7cm}
	\begin{alertblock}{
		\begin{center}
			\textbf{Status of my Master Thesis}
		\end{center}}
		\vspace*{10pt}
		\begin{center}\small
		Michael Reichmann
		\end{center}\normalsize
	\end{alertblock}
\end{frame}
% ============================
% TABLE OF CONTENTS
% ============================
\begin{frame}[allowframebreaks]
	\frametitle{Table of contents}
	\tableofcontents   % [pausesections]
\end{frame}

% ====================================================================================
% HARDWARE
% ====================================================================================
\section{Hardware}
% ============================
% THE PIXEL DETECTOR
% ============================
\subsection{The Silicon Pixel Detector}
\begin{frame}
	\frametitle{The Silicon Pixel Detector}
	\begin{minipage}{4cm}
		\begin{center}
			\begin{figure}
				\includegraphics<1>[height=4cm]{Pics/ROC1.png}
				\caption{dimensions of the ROC}
			\end{figure}
		\end{center}
	\end{minipage}
	\hspace*{2pt}
	\begin{minipage}{6.5cm}
		\begin{itemize}
			\item silicon sensor electrically connected to a read out chip (ROC) via Indium bump bonds
			\item each silicon pixel connected to pixel unit cell (PUC) on the ROC
			\item $4160$ pixels in $26$ double columns and $80$ rows
			\item size of a pixel: $150\,\upmu$m $\times$ $100\,\upmu$m
			\item size of the sensor: $7.8\,$mm $\times$ $8.0\,$mm (area: $62.4\,$mm$^{2}$) 
		\end{itemize}
	\end{minipage}\no\s
\end{frame}
% ============================
% THE SILICON PLANE
% ============================
\subsection{The Silicon Plane}
\begin{frame}
	\frametitle{The Silicon Plane}
	\begin{itemize}
		\item stable mounting framework for a single pixel detector
		\item amplifying circuit
		\item fast-OR differential LEMO output for pixel hits
		\item no signal termination
		\item bias voltage LEMO connector for the sensors
		\item 48 pin male connector on the back end
	\end{itemize}
\end{frame}
% ============================
% THE OLD ANALOGUE TELESCOPE
% ============================
\subsection{The Analogue Telescope}
\begin{frame}
	\frametitle{The Analogue Telescope}
	TODO: include picture of the telescope 
	\begin{itemize}
		\item main frame with six female 48 pin connectors for the planes
		\item one male connector to a test-board (mainly built for an analogue test-board)
		\item main operation mode: two planes in the front, two in the back to put a target in between
		\item two token jumper for the two empty connectors
		\item framework to hold the telescope and the target
	\end{itemize}
\end{frame}
% ============================
% THE NEW TELESCOPE
% ============================
\subsection{The New Telescope}\begin{frame}TODO\end{frame}
% ============================
% THE DIGITAL TESTBOARD
% ============================
\subsection{The Digital Test-Board (DTB)}
\begin{frame}
	\frametitle{The Digital Test-Board}
	TODO: include picture 
	\begin{itemize}
		\item FPGA including soft Token Bit Manager (TBM) emulator
		\item two digital and two analogue LEMO outputs which can be assigned to various internal signals
		\item USB and Ethernet outlet to connect to a computer
		\item connector to ROC
	\end{itemize}
\end{frame}


% ====================================================================================
% SOFTWARE
% ====================================================================================
\section{Software}
\begin{frame}TODO\end{frame}
\subsection{pXar}
\subsection{Python Command Line Interface (CLI)}
\subsection{EUDAQ}
\subsection{Keithley Current}
\subsection{EUDAQ Log Reader}
\subsection{High Voltage Client}
\subsection{Tracking}

% ====================================================================================
% EXPERIMENTS AND MEASUREMENTS
% ====================================================================================
\section{Experiments \& Measurements}
\begin{frame}TODO\end{frame}
\subsection{DTB Readout of the Analogue Planes}
\subsection{Study of the Fast-Or Signals}
\subsection{DESY Test-Beam}
\subsection{PSI Test-Beam}
\begin{frame}TODO\end{frame}

%==================================================================================================================================================
%===============================================SYNCHROTRON========================================================================================
%==================================================================================================================================================
% \section{Synchrotron}
% \subsection{Setup}
% \begin{frame}
% \begin{minipage}{4cm}
% \begin{center}
% % \includegraphics<1>[height=4cm]{Bilder/03}
% % \includegraphics<2>[height=4cm]{Bilder/04}
% % \includegraphics<3>[height=4cm]{Bilder/03}
% \end{center}
% \end{minipage}
% \hspace*{2pt}
% \begin{minipage}{6.5cm}
% \small\be\small F_R=F_L\ \rightarrow\ \frac{mv^2}{R}=qvB\ee
% \be \curvearrowright\ p=mv=qBR\ee
% \be \uptau=\frac{2\uppi R}{v}\ka \curvearrowright\ka \upomega_c=\frac{v}{R}=\frac{qB}{m_e\upgamma}\ee
% \begin{center}
% $\upomega_c-\z{cyclotron frequency}$
% \end{center}
% \end{minipage}\no\s
% \begin{overprint}
% \onslide<1>
% \bi \small
% 	\item constant radius $R$ and increasing magnetic Field $B$
% \ei
% \onslide<2>
% \bi \small
% 	\item constant radius $R$ and increasing magnetic Field $B$
% 	\item done with one or more preaccelerators: Cockroft-Walton $\rightarrow$ linac $\rightarrow$ pre-synchrotron $\rightarrow$ real synchrotron
% 	\bi	\item neccessary amount of energy to keep particles on a fixed circular orbit
% 		\item ratio of injected to maximum energy has to be small $\rightarrow$ higher acceptance $\curvearrowright$ gain in intensity\ei
% 	\item first accelerator called booster, last called main ring
% \ei
% \onslide<3>
% \bi\small
% 	\item magnets that keep particles on circular trajectory (bending magnets): dipole magnets
% 	\item focussing magnets: quadrupole magnets
% 	\item accelerating tube: radio frequency (RF) cavity
% 	\item synchronized frequency so that acceleration every time passing a cavity
% 	\item no continous beam $\rightarrow$ bunching in a linac ($\sim 1\,$mm bunch size)
% \ei 
% \end{overprint}
% \end{frame}
%==================================================================================================================================================
%===============================================SYNCHROTRON========================================================================================
%==================================================================================================================================================
% \subsection{Beam stability}
% \begin{frame}
% \frametitle{Beam stability}
% \bi
% 	\item velocity dispersion of the particles in the bunch
% 	\item Synchrotron can work if the not exactly synchronous particles 
% 	\bi \item tend to converge \item oscillate around the center \ei
% 	\item no stable acceleration if motion diverges
% \ei
% \end{frame}
% %==================================================================================================================================================
% %===============================================BETATRON OSCILLATION===============================================================================
% %==================================================================================================================================================
% \begin{frame}
% \frametitle{a) Betatron oscillation}
% \begin{minipage}{5.5cm}
% \begin{center}
% \includegraphics[width=4cm]{Bilder/13}
% \end{center}
% \end{minipage}
% \hspace*{2pt}
% \begin{minipage}{5.5cm}
% \begin{center}
% \includegraphics[width=4cm]{Bilder/14}
% \end{center}
% \end{minipage}\no\s
% \bi \small
% 	\item oscillations transverse to the beam
% 	\item controlled by quadrupole magnets
% 	\item one quadrupole: focussing only in one transverse direction, defocussing in the other
% 	\item two quadropoles (one turned by 90�): focussing in both directions\as
% 	\item $\rightarrow$ Synchrotron: interchanging dipole and quadrupole magnets in a FDFDFD... structure
% \ei
% \end{frame}
% %==================================================================================================================================================
% %===============================================PHASE STABILITY====================================================================================
% %==================================================================================================================================================
% \begin{frame}
% \frametitle{b) Phase stability}
% \begin{minipage}{4cm}
% \begin{center}
% \includegraphics[width=4cm]{Bilder/05}
% \end{center}
% \end{minipage}
% \hspace*{2pt}
% \begin{minipage}{7.5cm}
% \bi \small
% 	\item longitudinal oscillation synchronized with applied RF-voltage: synchrotron oscillation
% 	\item circular motion of $O$ is synchronous with RF (always the same acceleration) $\rightarrow$ synchronous phase
% \ei
% \end{minipage}\no\s
% \bi \small
% 	\item too early arriving particle $E$: 
% 	\bi \item gets more acceleration (more $E$) $\rightarrow$ more effective mass $\curvearrowright$ with $v\approx c$ bigger Radius
% 		\item $\uptau=\frac{R}{v}$ increases $\rightarrow$ arrive later next turn at $E'$\ei
% 	\item too late arriving particle $L$ gets less $E$ $\rightarrow$ smaller Radius $\rightarrow$ arrive earlier at $L'$
% \ei
% $\rightarrow$ convergation to synchronous phase
% 
% 
% \end{frame}
% %==================================================================================================================================================
% %===============================================RESULTS============================================================================================
% %==================================================================================================================================================
% \subsection{Synchrotron radiation}
% \begin{frame}
% \frametitle{Synchrotron radiation}
% \bi 
% 	%\item attenuates both betatron and sychrotron oscillation $\rightarrow$ also stabilizes the beam
% 	\item particle emits light when a electromagnetic field exerts a force on it (e.g. bended by a magnet)
% 	\item energy cannot be made arbitrary large proportional to the radius
% 	\item energy loss too large to be compensated
% 	\be \Updelta E=\frac{4\uppi\hbar c \upbeta^3\upgamma^4}{3R}\ee
% 	\be \Updelta E[\z{MeV}] \sim \frac{0.0085(E[\z{GeV}])^2}{R[\z{m}]}\ka \z{for electrons}\ee
% 	\item at LEP: $\Updelta E=3.2\,$GeV/turn ($R=2804\,$m $E=100\,$GeV)
% 	\item proton is 2000 times heavier $\curvearrowright$ loss is $2000^4\sim 10^{13}$ times smaller 
% 	\item $\Updelta E=6.7\,$keV/turn at LHC
% \ei
% \end{frame}
% %==================================================================================================================================================
% %===============================================APPLICATIONS=======================================================================================
% %==================================================================================================================================================
% \subsection{Applications}
% \begin{frame}
% \begin{minipage}{5.5cm}
% X-ray microscopy
% \end{minipage}
% \hspace*{2pt}
% \begin{minipage}{5cm}
% \begin{center}
% \includegraphics[width=5cm]{Bilder/15}
% \end{center}
% \end{minipage}\no\s
% \begin{minipage}{5.5cm}
% determine the origin of chrystals
% \bi  \item origin of blood diamonds \ei
% \end{minipage}
% \hspace*{2pt}
% \begin{minipage}{5cm}
% \begin{center}
% \includegraphics[width=5cm]{Bilder/16}
% \end{center}
% \end{minipage}\no\s
% \end{frame}
% %==================================================================================================================================================
% %===============================================APPLICATIONS=======================================================================================
% %==================================================================================================================================================
% \begin{frame}
% \begin{minipage}{5.5cm}
% Micromachining 
% \bi \item production of tiny machine parts  \ei
% \end{minipage}
% \hspace*{2pt}
% \begin{minipage}{5cm}
% \begin{center}
% \includegraphics[width=5cm]{Bilder/17}
% \end{center}
% \end{minipage}\no\s
% \begin{minipage}{5.5cm}
% Medicine and Pharmaceuticals
% \bi\item study and model influenza virus proteins \ei
% \end{minipage}
% \hspace*{2pt}
% \begin{minipage}{5cm}
% \begin{center}
% \includegraphics[width=5cm]{Bilder/18}
% \end{center}
% \end{minipage}\no\s
% \end{frame}
% %==================================================================================================================================================
% %===============================================XFEL=========================================================================================
% %==================================================================================================================================================
% \section{Free Electron Laser}
% \subsection{Setup}
% \begin{frame}
% \bi 
% 	\item emittance of synchrotron radiation (SR) if electromagnetic fields exerts force on a charged particle
% 	\item use external electromagnetic fields to emit even more SR
% 	\item particle continously has to lose energy into SR $\rightarrow$ field must have certain frequency and phase
% \ei
% \pause
% \begin{center}
% \includegraphics[height=2.5cm]{Bilder/07}
% \end{center}
% \bi 
% 	\item generally: recycling and usage of spontaneous radiation for next emissions by using mirrors
% 	\item lack of suitable mirros in UV and X-Ray regimes
% \ei
% \end{frame}
% \begin{frame}
% SwissFEL facility with beam line tunnel experimental hall and infrastructure
% \begin{center}
% \includegraphics[width=12cm]{Bilder/06}
% \end{center}
% \end{frame}
% 
% %==================================================================================================================================================
% %===============================================WORKING PRINCIPLE I================================================================================
% %==================================================================================================================================================
% \subsection{Working principle}
% \begin{frame}
% \bi 
% 	\item propagation on a sinusoidal path with $v\simeq c$
% 	\item emittance of SR in a small cone in forward direction
% 	\item radiation from individual magnetic periods overlap $\rightarrow$ interference
% \ei
% \begin{center}
% \includegraphics[width=6cm]{Bilder/08}
% \end{center}
% \bi 
% 	\item SR faster then the electron
% 	\item electrons have to slip one radiation wavelength with respect to the faster electromagnetic field during one period of the undulator
% 	%\item $\curvearrowright$ amplification
% \ei
% \end{frame}
% 
% %==================================================================================================================================================
% %===============================================WORKING PRINCIPLE II===============================================================================
% %==================================================================================================================================================
% \begin{frame}
% \begin{center}
% \includegraphics[width=12cm]{Bilder/09}
% \end{center}
% \bi \small
% 	\item Interaction of the oscillating bunge and the electromagnetic field
% 	\item deceleration if electron and e/m-wave are in phase
% 	\item acceleration if off phase
% 	\item longitudinal fine structure called microbunching
% 	\item longitudinal distribution cut into equidistant slices
% 	%\item seperation length: wavelength of the emitted radiation
% 	\item more and more electrons start radiating in phase
% 	\item incresingly coherent superposition of the radiation 
% 	\item extremely short and intense X-ray flashes with the properties of laser light
% \ei
% \end{frame}
% %==================================================================================================================================================
% %===============================================APPLICATIONS I=====================================================================================
% %==================================================================================================================================================
% \subsection{Applications}
% \begin{frame}
% \frametitle{Deciphering the structure of biomelocules}
% \bi \item X-ray sources not strong enough to look at single molecules\ei
% \begin{center}
% \includegraphics[width=7cm]{Bilder/10}
% \end{center}
% \end{frame}
% %==================================================================================================================================================
% %===============================================APPLICATIONS II====================================================================================
% %==================================================================================================================================================
% \begin{frame}
% \frametitle{Filming chemical reactions}
% \bi \item flashes: less than 0.1 trillionth of a second $\rightarrow$ snapshots without moving details becoming blurred\ei
% \begin{center}
% \includegraphics[width=7cm]{Bilder/11}
% \end{center}
% \end{frame}
% %==================================================================================================================================================
% %===============================================APPLICATIONS III===================================================================================
% %==================================================================================================================================================
% \begin{frame}
% \frametitle{Investigate extreme states of matter}
% \bi \item plasmas can be created that are as hot as the interiors of giant stars\ei
% \begin{center}
% \includegraphics[width=7cm]{Bilder/12}
% \end{center}
% \end{frame}



\end{document}